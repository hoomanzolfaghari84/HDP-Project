\documentclass[11pt, a4paper]{article}

%-----------------------------------------------------------
% PACKAGES
%-----------------------------------------------------------
\usepackage[utf8]{inputenc}           % For UTF-8 encoding
\usepackage[T1]{fontenc}
\usepackage{lmodern}
\usepackage{amsmath, amssymb, amsthm}   % AMS math packages
\usepackage{mathtools}                % Extra math symbols and tools
\usepackage{graphicx}                 % For including images
\usepackage{hyperref}                 % For hyperlinks
\usepackage{cleveref}                 % Smart referencing
\usepackage{enumitem}                 % Customize lists
\usepackage{geometry}                 % Adjust margins and paper size
\geometry{margin=1in}
\usepackage{color}
\usepackage{bm}                       % Bold math symbols

%-----------------------------------------------------------
% THEOREM ENVIRONMENTS
%-----------------------------------------------------------
\newtheorem{theorem}{Theorem}[section]
\newtheorem{lemma}[theorem]{Lemma}  % Number lemmas with theorems
\newtheorem{corollary}[theorem]{Corollary}
\newtheorem{proposition}[theorem]{Proposition}

\theoremstyle{definition}
\newtheorem{definition}[theorem]{Definition}
\newtheorem{example}[theorem]{Example}

\theoremstyle{remark}
\newtheorem{remark}[theorem]{Remark}

%-----------------------------------------------------------
% CUSTOM MACROS
%-----------------------------------------------------------
\newcommand{\R}{\mathbb{R}}
\newcommand{\E}{\mathbb{E}}
\newcommand{\Prob}{\mathbb{P}}
\newcommand{\empnorm}[1]{\|#1\|_{\mu_n}}
\newcommand{\inner}[2]{\langle #1 , #2 \rangle }
\newcommand{\tr}{\operatorname{Tr}}
\newcommand{\bH}{\mathbf{H}}

%-----------------------------------------------------------
% TITLE INFORMATION
%-----------------------------------------------------------
\title{High-Dimensional Probability and the Neural Tangent Kernel}
\author{Your Name\thanks{Your Affiliation, email@example.com}}
\date{\today}

%-----------------------------------------------------------
% DOCUMENT START
%-----------------------------------------------------------
\begin{document}
	
	\maketitle
	
	
	\begin{abstract}
		
	\end{abstract}
	
%	\tableofcontents
	
	We analyze the convergence and generalization of a coupling based normalizing flow model.
	
	\section{Preliminaries}
	
	\subsection{Hypothesis set}
	
	Define \(\mathcal{F} = \{f :\R^d \to \R^d\}\) as the set of all \textit{Non-Volume Preserving Coupling-Based Normalizing Flows}. And define the \textit{Realization Function} \(F:\R^P \to \mathcal{F}\). Now we have for an \(f_\theta := F(\theta)\) the following structure:
	
	Denote the permutation by a matrix $Q \in \mathbb{R}^{d \times d}$ (an orthonormal permutation matrix), 
	so that
	\[
	\tilde{x} = Q x.
	\]
	We then split $\tilde{x}$ into two parts, 
	\[
	\tilde{x} 
	\;=\; (\tilde{x}_1, \,\tilde{x}_2)
	\quad \text{with} \quad
	\tilde{x}_1 \in \mathbb{R}^{d'}, 
	\;\tilde{x}_2 \in \mathbb{R}^{d - d'}.
	\]
	An \emph{affine coupling} transform $(\tilde{x}_1, \tilde{x}_2) \mapsto (\tilde{y}_1, \tilde{y}_2)$ 
	is defined by
	\[
	\tilde{y}_1 
	\;=\; \tilde{x}_1,
	\qquad
	\tilde{y}_2 
	\;=\; \tilde{x}_2 \;\odot\; \exp\bigl[s_{\theta_1}(\tilde{x}_1)\bigr]
	\;+\; t_{\theta_2}(\tilde{x}_1),
	\]
	where 
	\[
	s_{\theta_1}: \mathbb{R}^{d'} \to \mathbb{R}^{d-d'}, 
	\quad
	t_{\theta_2}: \mathbb{R}^{d'} \to \mathbb{R}^{d-d'}.
%	\quad \theta = \begin{bmatrix}
%		\theta_1 \\
%		\theta_2
%	\end{bmatrix} \in \R^P
	\]
	are fully connected neural networks parameterized each by a subspace of $\R^P$. 
%	Finally we invert the permutation:
%	\[
%	y \;=\; Q^{-1}\,\tilde{y} 
%	\quad \in \mathbb{R}^D.
%	\]
	Thus, the overall transformation is 
	\[
	T_{\theta_1,\theta_2}(x) \;=\;
	\Bigl(
	\tilde{x}_1,\;
	\tilde{x}_2 \odot \exp[s_{\theta_1}(\tilde{x}_1)] \;+\; t_{\theta_2}(\tilde{x}_1)
	\Bigr)^\top
	\quad\text{where}\quad 
	\tilde{x} = Q x.
	\].
	
	The Tangent Kernel of \(f_\theta\)
	
	A \textit{coupling based normalizing flow model} of depth \(L\) is therefore comprised by:
	\[
	f_\theta (x) = T^{(L)} \circ  T^{(L-1)} \circ \cdots \circ  T^{(1)} (x) 
	 \]
	 , and we denote \(\theta = (\theta_1^s , \theta_1^t, \theta_2^s , \theta_2^t , \cdots ,\theta_L^s , \theta_L^t) \in \R^P \)
	
	\paragraph{Jacobian Determinant.}
	Because $Q$ is a permutation matrix with $\lvert \det Q \rvert = 1$, its contribution to the Jacobian determinant is 1.
	The remaining (affine coupling) block is
	\[
	\frac{\partial (\tilde{y}_1, \tilde{y}_2)}{\partial (\tilde{x}_1, \tilde{x}_2)}
	\;=\;
	\begin{pmatrix}
		I_{d} & 0 \\
		* & \mathrm{diag}\!\bigl(\exp[s_\theta(\tilde{x}_1)]\bigr)
	\end{pmatrix},
	\]
	so the absolute Jacobian determinant is
	\[
	\bigl|\det \nabla_x f_\theta(x)\bigr|
	\;=\;
	\prod_{j=1}^{\,d-d'} \exp\bigl[s_{\theta,j}(\tilde{x}_1)\bigr]
	\;=\;
	\exp\!\Bigl(\sum_{j=1}^{\,d-d'} s_{\theta,j}(\tilde{x}_1)\Bigr).
	\]
	\subsection{Loss Criterion}
	For a CPNF \(z = f_{\theta}(x)\),
	with the base density \( p_Z(z) \). By the change-of-variables formula, the model density is given by
	\[
	\hat{p}_X(\mathbf{x}) = p_Z\bigl(f_{\boldsymbol{\theta}}(\mathbf{x})\bigr) \, \Bigl|\det J(f_\theta,\mathbf{x}) \Bigr|.
	\]
	and minimizing the KL-divergence between \(\mu=p_X\)and \(\hat{p}_X\), will be equivalent to minimizing the expectation of the negative log-likelihood of \(\hat{p}_X\) with respect to \(p_X\) and is approximated by the empirical risk as below:
	\[
	\mathcal{L}(\mathbf{x}; f) = -\log\hat{p}_X(\mathbf{x}) = -\log p_Z\bigl(f(\mathbf{x})\bigr) - \log \Bigl|\det J(f,\mathbf{x}) \Bigr|
	\]
	\paragraph{Negative Log-Likelihood.}
	We choose a standard normal base distribution $p_z(z) = \mathcal{N}(z \mid 0, I)$ on $\mathbb{R}^D$. 
	Then for $y = f_\theta(x)$, 
	the model's density is 
	\[
	p_\theta(x)
	\;=\;
	p_Z\bigl(y\bigr)
	\,\bigl|\det \nabla_x f_\theta(x)\bigr|.
	\]
	Hence the negative log-likelihood is
	\[
	-\log p_\theta(x)
	\;=\;
	-\log p_z\!\bigl(y\bigr)
	\;-\; 
	\log \bigl|\det \nabla_x f_\theta(x)\bigr|.
	\]
	If $p_z$ is standard Gaussian, 
	\[
	-\log p_z(y)
	\;=\;
	\frac12 \|y\|^2 \;+\; \frac{D}{2}\,\log (2\pi)
	\quad (\text{ignoring constants in } \theta).
	\]
	Using the above determinant,
	\[
	-\log \bigl|\det \nabla_x f_\theta(x)\bigr|
	\;=\;
	-\,\sum_{j=1}^{\,D-d} s_{\theta,j}\!\bigl(\tilde{x}_1\bigr).
	\]
	Thus the single-sample loss becomes
	\[
	\ell(\theta; x)
	\;=\;
	-\log p_\theta(x)
	\;=\;
	\frac12 \|f_\theta(x)\|^2 
	\;-\;
	\sum_{j=1}^{\,D-d} s_{\theta,j}\!\bigl(\tilde{x}_1\bigr)
	\;+\;
	\text{const.},
	\]
	and the \emph{training objective} for a dataset $\{x_i\}_{i=1}^N$ is 
	\[
	\mathcal{L}(\theta)
	\;=\;
	\frac{1}{N} \sum_{i=1}^N
	\Bigl[
	- \log p_z\!\bigl(f_\theta(x_i)\bigr)
	\;-\;
	\log \bigl|\det \nabla f_\theta(x_i)\bigr|
	\Bigr] 	= \frac1N \sum_{i=1}^{N} \ell(\theta;x)
	\].

	\begin{remark}
		We might be able to show that NLL is equivalent to some supervised loss for some unknown optimal \(f^*\), that \(Z = f^*(X)\)
	\end{remark}

	\subsection{Probability Space}
	
	Denote the probability space \((\Omega,\Sigma,\Prob)\), where \(X: \Omega \to \R^d\) has the unknown distribution \(\mu \in \mathcal{M}(\R^d)\) and \(Z \sim N(0,I_d)\). Also define the empirical distribution \(\mu_N\) for our dataset \(S=\{X^{(i)}\}_{i=1}^N\), as \(\mu_N(A) = \frac{1}{N}\sum_{i=1}^{N} \mathbf{1}_{X_i \in A}\), where \(X_i\) are independent copies of \(X\).\\
	
	The ultimate learning objective is to minimize the \textit{statistical risk}:
	\[
	R(f):=\E_\mu[\ell(\theta;X)]
	\]
	, and the \textit{empirical risk} is: 
	\[
	\hat{R}_S(f):=\frac{1}{|S|}\sum_{x\in S} \ell(\theta;x)
	\]
	
%	Define the semi-norm and bi-linear form below:
%	\[
%	\empnorm{f} = \E_{\mu_n}[f], \quad \inner{f}{g}_{\mu_n} = \E_{\mu_{n}}[f^\top g]
%	\]
%	
%	
	
%	Define the \textbf{Jacobean Functional} \(J:\mathcal{F}\times \mathbb{R}^d \to \mathbb{R}^{d\times d}\) as:
%	\[
%	J(f,x):=
%	\begin{pmatrix}
%		\frac{\partial f_1}{\partial x_1}(x) & \frac{\partial f_1}{\partial x_2}(x) & \cdots & \frac{\partial f_1}{\partial x_d}(x) \\
%		\frac{\partial f_2}{\partial x_1}(x) & \frac{\partial f_2}{\partial x_2}(x) & \cdots & \frac{\partial f_2}{\partial x_d}(x) \\
%		\vdots & \vdots & \ddots & \vdots \\
%		\frac{\partial f_d}{\partial x_1}(x) & \frac{\partial f_d}{\partial x_2}(x) & \cdots & \frac{\partial f_d}{\partial x_d}(x)
%	\end{pmatrix}
%	= 	\begin{pmatrix}
%	    \nabla f_1(x)^\top \\
%		\nabla f_2(x)^\top \\
%		\vdots \\
%		\nabla f_d(x)^\top
%	\end{pmatrix} = \nabla f(x)^\top
%	\],
%	 and the functions \(J_{\mu_n}(f) := \frac{1}{N} \sum_{i=1}^{N} J(f,X_i) \) and \(J(f):=h \implies h(x)=J(f,x)\)

	
	\section{The Neural Tangent Kernel}
	
	 H: Maybe we should see the whole \(p_\theta\) as the network ?\\
	 
	 
	 
	
	First we compute gradient of a transform \(T^{(i)}=T_{\theta_1,\theta_2}\) as:
	
	\[
	\nabla_{(\theta_1,\theta_2)}\,T_{\theta_1,\theta_2}(x)
	=
	\begin{bmatrix}
		\mathbf{0}_{\,\dim(\tilde{x}_1)\times \dim(\theta_1)} 
		& 
		\mathbf{0}_{\,\dim(\tilde{x}_1)\times \dim(\theta_2)} 
		\\[6pt]
		\tilde{x}_2\,\odot\,\exp\bigl[s_{\theta_1}(\tilde{x}_1)\bigr]
		\,\odot\,
		\nabla_{\theta_1}\,s_{\theta_1}\bigl(\tilde{x}_1\bigr)
		&
		\nabla_{\theta_2}\,t_{\theta_2}\bigl(\tilde{x}_1\bigr)
	\end{bmatrix} := \partial T
	\]
	
	and so we have
	\(
	\nabla_\theta T_{\theta_1,\theta_2}(x) = (
		0 ,
		\cdots ,
		\partial T ,
		\cdots,
		0
) \in \R^{P \times d}
	\).\\
	

	
	We can define the \textit{Tangent Kernel} of a CBNF \(f_\theta\) as \(H(t)\), which is an \(N\times N\) positive semi-definite matrix whose \((i,j)\)-th entry is \(\inner{\frac{\partial f(\theta(t),x_i)}{\partial \theta}}{\frac{\partial f(\theta(t),x_j)}{\partial \theta}}\).
	
	
	
	
	For our \textbf{Training Cost Functional} \(\mathcal{L}(\theta)\), we have:
	\[
	\nabla_\theta \mathcal{L}(\theta) = \nabla_\theta f
	\]
	\[
	C^{\mu_N}(f) = \| -\log p_Z\bigl(f \bigr) - \log \Bigl|\det J(f) \Bigr| \|_{\mu_N} =\frac{1}{N} \sum_{i=1}^{N}  -\log p_Z\bigl(f(X_i)\bigr) - \log \Bigl|\det J(f,X_i)\Bigr|
	\]
	, which is \(\hat{R}_S\), for our \(S\).
	Now we take the (functional) derivative of this with respect to a function \(f_\theta\), denoted by:
	\[
	\partial^{\mu_N}_{f} C\bigr|_{f_\theta} = \empnorm{ 
		f - \bigl[ \tr\left( J(f)^{-1} \frac{\partial J(f)}{\partial f_1} \right) \; \cdots \; \tr\left( J(f)^{-1} \frac{\partial J(f)}{\partial f_d} \right) \bigr]^\top
	}  
	\]
	
		This is a point in \(\mathcal{F}^*\) and so is a function of \(f \in \mathcal{F}\) to \(\R\).,and we can denote it by \(\phi_\theta:\mathcal{F}\to \R\).
	
	\subsection{Deriving the Kernel}
	So denoting \(f_\theta := F(\theta)\) the parameters change as below in gradient descent:
	\[
	\frac{d \theta_p}{dt} =  - \eta \partial_{\theta_p} C\circ F(\theta) = -\eta \ \partial^{\mu_N}_{f}C\bigr|_{f_\theta} (\partial_{\theta_p} F(\theta))  = 
- \eta \ \phi_\theta (\partial_{\theta_p} F(\theta))
	\]
, giving us the gradient form:

 \[\nabla_\theta C\circ F(\theta)  = [ \phi_\theta(\partial_{\theta_1}F(\theta)) \; \cdots \phi_\theta(\partial_{\theta_P}F(\theta)) ]^\top 
 ,\quad
 \frac{d \theta}{d t} = - \eta \ \nabla_\theta C\circ F(\theta)
 \]

Defining the Neural Tangent Kernel \(\bH(\theta)\)
We want to solve the differential equation below:
\[
\partial_{t} f_{\theta(t)} = - \frac{1}{N} \sum_{i=1}^{N} 
\]



From the blog:
	\[
\frac{df(\mathbf{x}';\theta)}{dt} 
= \frac{df(\mathbf{x}';\theta)}{d\theta}\frac{d\theta}{dt}
= -\frac{1}{N} \sum_{i=1}^N \color{blue}{\underbrace{\nabla_\theta f(\mathbf{x};\theta)^\top \nabla_\theta f(\mathbf{x}^{(i)}; \theta)}_\text{Neural Tangent Kernel}    } \color{black}{\nabla_f \ell(f, y^{(i)})}
	\]
	
	\[
	\frac{df(\mathbf{x}';\theta)}{dt} 
	= \frac{df(\mathbf{x}';\theta)}{d\theta}\frac{d\theta}{dt}
	= -\frac{\eta}{N} \sum_{i=1}^N \nabla_\theta f(\mathbf{x}';\theta)^\top \nabla_\theta f(\mathbf{x}^{(i)}; \theta) \color{black}{\nabla_f \ell(f, y^{(i)})}
	\]
	
	\section{Random Process View}
	
	It is well-known that Neural Networks in the infinite-width output a Gaussian process at their Gaussian initialization.\\
	But what happens in training to a Neural Network.
\textbf{In training} the cost function follows a Gaussian process, indexed by the space \(\R^P\) of parameters. we can create some empirical processes according to the empirical risk and statistical risk indexed by the hypothesis \(\mathcal{F}\) or \(\R^P\).
 
We have three probability measures so far: \(N(0,I), \mu, \mu_N\).
\begin{definition}{\textbf{Learning Processes}}\\
Take the functions \(C:\mathcal{F}\to \R\) and \(R:= C\circ F : \R^P \to \R\), we can use the laws \(\mu\) and \(\mu_N\) such that:
\begin{itemize}
	\item \textbf{Function Processes}: The random variables \(f(X)\) or \(F(\theta)(X)\), respectively indexed on \(\mathcal{F}\) and \(\R^P\), with \(X\sim\mu\) or \(\mu_N\).
	
	\item \textbf{Empirical Processes}: The functions \(\mathbb{G}_N(f):=\sqrt{N}(\mu_N - \mu)f\) and  \(\mathbb{G^\theta}_N(\theta):=\sqrt{N}(\mu_N - \mu)F(\theta)\), respectively indexed on \(\mathcal{F}\) and \(\R^P\), with \(X\sim\mu\) or \(\mu_N\).
	
	\item \textbf{Training Processes}: The functions \(C(f)\) and \(C \circ F(f)\), when \(S=\{X_i\}_{i=1}^N \sim \mu^N\) or maybe even \(S\sim \mu_N^N\) can be useful.
	
	
\end{itemize}
\end{definition}

We shall analyze properties of this random process. For example the ultimate goad of machine learning would be characterized as \(\E[\sup_{\theta} R(\theta)]\), which is the statistical risk for the objective learning concept. Our empirical risk is also characterized by \(\E\sup_{\theta} \hat{R}(\theta)\)

What is the covariance function of these processes ? is it NTK ?
 Can we apply the results of the books on Random Processes to these ?

	
\section{Sub-Gaussian Investigation}
	
		
	
	
	%-----------------------------------------------------------
	% BIBLIOGRAPHY
	%-----------------------------------------------------------
	\bibliographystyle{plain}
	\bibliography{references}  % Make sure to create a 'references.bib' file
	
	\appendix
	
	\section{Additional Proofs}
	\label{sec:appendix}
	Additional technical proofs, auxiliary lemmas, or numerical experiments may be included here.
	
\end{document}
